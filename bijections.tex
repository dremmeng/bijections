\documentclass[10pt, oneside]{article}
\usepackage{amsmath, amsthm, amssymb, calrsfs, wasysym, verbatim, bbm, color, graphics, geometry, cite}
\geometry{tmargin=.75in, bmargin=.75in, lmargin=.75in, rmargin = .75in}
















\newcommand{\R}{\mathbb{R}}
\newcommand{\C}{\mathbb{C}}
\newcommand{\Z}{\mathbb{Z}}
\newcommand{\N}{\mathbb{N}}
\newcommand{\Q}{\mathbb{Q}}
\newcommand{\Cdot}{\boldsymbol{\cdot}}
















\newtheorem{thm}{Theorem}
\newtheorem{defn}{Definition}
\newtheorem{conv}{Convention}
\newtheorem{rem}{Remark}
\newtheorem{lem}{Lemma}
\newtheorem{cor}{Corollary}
\newtheorem{example}{Example}
\newtheorem{exe}{Exercise}
\newtheorem{conjecture}{Conjecture}
\newtheorem{remark}{Remark}
\title{Notes on the Structure $C^k$ Functions}
\author{[Drew Remmenga]}
















\begin{document}
















\maketitle
\begin{abstract}
\end{abstract}
\section*{$C^k$ Smooth Functions}
  A function with $k$ continuous derivatives on a domain $X$ is denoted by $C^k(X)$ \cite{Lang1999}
  There is an ordering:
  \begin{align*}
     C^0 \subset C^1 \subset \dots \subset C^{\omega} \subset C^{\infty}
  \end{align*}
  Each $C^k$ satisfies the axioms of a ring.
  \begin{proof}
     We take: $f, g, h \in C^k$.
     Closure under addition: $(f+g) \in C^k$ because differentiation is linear:
     \begin{align*}
        \partial^\alpha (f+g) = \partial^\alpha f + \partial^\alpha g
     \end{align*}
     Closure under multiplication: $f \cdot g \in C^k$ by Leibniz rule we know:
     \begin{align*}
        \partial^\alpha = \sum_{\beta \leq \alpha} {\alpha \choose \beta} (\partial^{\beta} f)(\partial^{\alpha - \beta}g)
     \end{align*}
     which remains $C^k$ continuous.
     Associativity under Addition and Multiplication:
     \begin{align*}
        (f+g)+h = f+(g+h), (f\cdot g)\cdot h = f\cdot (g \cdot h)
     \end{align*}
     Commutativity of Addition and Multiplication:
     \begin{align*}
        f+g=g+f, f\cdot g = g\cdot h
     \end{align*}
     Additive identity: given by the zero function $0(X) = 0$, $0 \in C^k$ and satisfies:
     \begin{align*}
        f+0=f, \forall f \in C^k
     \end{align*}
     Additive inverses: $\forall f \in C^k$ the function $-f \in C^k$ and satisfies:
     \begin{align*}
        f+(-f) = 0
     \end{align*}
     Multiplicative identity: the constant function $1(X) = 1 \in C^k$ and satisfies:
     \begin{align*}
        1 \cdot f = f, \forall f \in C^k
     \end{align*}
     Distributivity of Multiplication over Addition:
     \begin{align*}
        f \cdot (g + h) = f \cdot g + f \cdot h
     \end{align*}
  \end{proof}
  There is an additional structure of $C^k$ functions. Namely we can compose them with each other. Denote $\circ$ composition of functions $f(g(X))$ for $f,g \in C^k$ when the domain $X=\R^d$
  This is a semigroup under this operation.
  \begin{proof}
     Take $f,g \in C^k$ as before. The space is closed under the operation: follows from the chain rule and the fact that derivatives of $f$ and $g$ up to $k$ are continuous.
     Associativity: inherited from function composition.
     \begin{align*}
        (f \circ g) \circ h = f \circ (g \circ h)
     \end{align*}
     An identity element exists $\text{id}(x) = x \in C^k$ and satisfies $f \circ \text{id} = \text{id} \circ f = f$.
  \end{proof}
\section*{Invertible $C^k$ Functions}
  If we restrict $C^k$ to the set of invertible functions we can satisfy all the group axioms under $\circ$.
  Take $\text{Diff}^k$ as the set of invertible $C^k$ functions. Then there is an ordering:
  \begin{align*}
     \begin{array}{cccccc}
        C^0 & \supset & C^1 & \supset & \cdots & \supset C^\infty \\
        \cup & & \cup & & & \cup \\
        \text{Diff}^0 & \supset & \text{Diff}^1 & \supset & \cdots & \supset \text{Diff}^\infty
\end{array}
  \end{align*}
  Then the only missing property to satisfy the group axioms was the existence of inverses and by construction now for a function $f$ we have the necessary inverse $f^{-1}$ such that:
  \begin{align*}
     f \circ f^{-1} = \text{id}
  \end{align*}
  Indeed $\text{Diff}^k$ is an infinite dimensional Lie Algebra and it is perfect.\cite{Banyaga1997}. We can imagine a forgetful functor inspired by representation theory and the $\text{Res}^G_H$ or the restriction from a group $G$ to a subgroup $H$.
  Define the functor $F: C^k \to \text{Diff}^k$. We can imagine a left adjoint functor to rebuild $C^k$ from $\text{Diff}^k$ through the $+$ and $\cdot$ operations on elements $f_1 \dots f_n \in \text{Diff}^k$. Take the functor $G: \text{Diff}^k \to C^k$.
  Indeed it may be possible to build the functors $F$ and $G$ out of elements $f_1 \dots f_n$ in the respective algebras using the ring and semi-group operations; at least within $\R^d$.
\bibliographystyle{plain}  % or another style like alpha, unsrt, etc.
\bibliography{references.bib}  % the name of the .bib file
\end{document}